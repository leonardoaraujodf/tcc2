\begin{resumo}[Abstract]
 \begin{otherlanguage*}{english}
Energy demand in the world is growing and the use of alternative energy sources has been highlighted 
in the national and international context, especially in wind energy, which has been the target of 
public and private investments.
In Brazil, this energy source plays a relevant role in its electric matrix and there is a growing number 
of studies of access to wind farms to the electric power system.
Due to the variability of the power produced by these sources, aspects that involve the quality of 
electric power are studied for the operation of these systems, using power converters, coupling 
filters, and control systems.
The objective of this work is the investigation through computational simulations and laboratory 
tests of the coupling of wind units to the electric grid.
It makes use of a software computational model to validate the projected physical components and 
system control algorithms and a laboratory bench using a three-phase inverter to certify the model 
operation.
The computer simulations conducted in software demonstrate that the voltage in the direct current 
bus is fixed at a certain value by the implemented control loops performing the power flow control.
With laboratory tests, it is possible to observe the operation of the various subsystems that make 
up the three-phase inverter.
The contribution of this work is the setting up of a winding bench. This fact enables studies 
regarding the connection of wind units and other systems that employ three-phase inverters to the 
electrical grid.

   \vspace{\onelineskip}
 
   \noindent 
   \textbf{Key-words}: three-phase inverter, wind energy, electric power system, power quality, power converters, SPWM, boost converter.
 \end{otherlanguage*}
\end{resumo}
