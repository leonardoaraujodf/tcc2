\begin{resumo}[Abstract]
 \begin{otherlanguage*}{english}
  There is a growing demand for energy in the world, and the use of alternative energy sources has been rough in the international context, especially wind energy, which has been the target of constant investments from both government and private organizations.
  In Brazil, this energy source plays an important role in its energy matrix and it is of special interest to study the wind coupling units to electric power systems.
  Due to the variable aspect of power supply from the source in question, the area of power quality is an object of study for the operation of these systems, making use of the power converters design, coupling filters and, control and synchronization algorithms.
  Thus, this work objective was computer simulation and elaboration of a workbench for study and investigation of wind coupling units to the electric grid, to present the design theory of three-phase power inverters, coupling LCL-filters, current control loops, continuous voltage control loops, and synchronization loop.
  A computer simulation was made in the PSIM software and satisfactory results were obtained for the inverter operation regarding the control of the continuous voltage bus through the injection or reception of power to or from the power grid.
  Soon after, practical experiments were performed with the bench, obtaining convincing operating results for the various subsystems that make up the three-phase inverter. The bench will also serve as a basis for future research projects and studies of the coupling of wind units to the electricity grid.

   \vspace{\onelineskip}
 
   \noindent 
   \textbf{Key-words}: three-phase inverter. wind energy. electric power system. power quality. power converters. SPWM. boost converter.
 \end{otherlanguage*}
\end{resumo}
