\chapter{Resultados}

\section{Resultados de Simulação Computacional}

\subsection{Inversor isolado da rede elétrica}

\subsection{Inversor como STATCOM}

\subsection{Inversor conectado ao conversor \textit{boost}}

Nesta etapa de simulação, o objetivo foi verificar se o inversor trifásico, juntamente com sua malha de controle, conseguiam realizar transmissão de potência ativa do barramento CC para a rede elétrica. 
Para isto, conectou-se o do barramento de CC à um conversor do tipo \textit{boost}, que injetava corrente no barramento.
O inversor, de forma a manter a tensão no barramento CC constante, realizava as iterações em seu algorítmo de controle de forma a injetar na rede elétrica a corrente recebida.
A Fig. \ref{fig:sim-circuito-inversor-boost} mostra o circuito utilizado.

\begin{figure}[!hbt]
	\begin{center}
    \includegraphics[width=\textwidth]{figuras/sim_figures/sistema_completo/montagem/sistema_completo.PNG}    \centering
    \caption{Circuito do inversor conectado com o conversor \textit{boost}, juntamente com os blocos de controle no software PSIM}
    \label{fig:sim-circuito-inversor-boost}
    \end{center}
\end{figure}

A tensão de referência definida para o barramento CC foi de 700 V. Desta forma, esperou-se que, apesar das pertubações a serem inseridas no sistema, este conseguisse estabilizar a tensão do barramento CC pŕoximo a tensão de referência após alguns segundos. 
Inicialmente o sistema foi inicializado sem nenhuma injeção de potência no barramento CC e permaneceu assim até os 4 segundos. Após, isso iniciou-se a injeção de potência através do conversor \textit{boost}, que foi programado para injetar 10 A de corrente.

Pode-se verificar na Fig. \ref{fig:sim-tensao-barramento} a estabilização de tensão no barramento CC. 
Antes dos 4 segundos, o sistema passou por um transitório mas conseguiu manter a tensão no barramento próximo de 700 V. 
Após isto, houve a inserção de corrente pelo \textit{boost}, que provocou uma pertubação, de forma a elevar a tensão sobre o \textit{link} capacitivo do barramento.
De forma a estabilizar esta tensão, o sistema começou a fornecer potência à rede elétrica de modo a manter a tensão CC próximo a seu valor de referência.

\begin{figure}[!hbt]
	\begin{center}
    \includegraphics[width=\textwidth]{figuras/sim_figures/sistema_completo/tensao_barramento.PNG}
    \caption{Tensão no barramento CC. A tensão de referência foi definida como 700 V. Inicialmente o sistema foi inicializado sem nenhuma injeção de potência no barramento, como pode-se verificar antes dos 4 segundos. Depois, acontece uma pertubação devido a inserção de potência, mas este estabiliza-se próximo a tensão de referência após alguns segundos}
    \label{fig:sim-tensao-barramento}
    \end{center}
\end{figure}

A Fig. \ref{fig:sim-tensao-saida-inversor} mostra as tensões sintetizadas na saída do inversor, após o filtro $L_2$.
Ao se verificar o expectro em frequência da tensão sintetizada na fase $V_A$, é possível ver que existem harmônicas perto da frequência de chaveamento, mas que não produzem alterações significativas na tensão de saída. 
Verifica-se ainda que houve a sintetização correta das tensões em cada fase.

Já a Fig. \ref{fig:sim-corrente-saida-inversor} mostra as correntes geradas após a inserção de potência pelo \textit{boost} no barramento CC.
O expectro em frequência da corrente produzida na fase A denominada $I_A$ mostra que existem harmônicas de baixa frequência, de quinta e sétima ordem, mas que não provocam alterações significativas na corrente produzida.
Ainda é possível verificar o envio correto das três correntes para a rede elétrica.

\begin{figure}[!hbt]
	\centering
	\begin{subfigure}[b]{\textwidth}
		\centering
		\includegraphics[width=\textwidth]{figuras/sim_figures/sistema_completo/tensao_saida_inversor_2.PNG}
		\caption{Tensão de saída sintetizada para a fase A}
   \end{subfigure}

   \begin{subfigure}[b]{\textwidth}
		\centering
		\includegraphics[width=\textwidth]{figuras/sim_figures/sistema_completo/tensao_saida_inversor_fft.PNG}
		\caption{Expectro em frequência da tensão sintetizada na fase A. Os harmônicos de alta frequência próximos a frequência de chaveamento de 5 kHz sofreram uma boa atenuação}
   \end{subfigure}

	\begin{subfigure}[b]{\textwidth}
		\includegraphics[width=\textwidth]{figuras/sim_figures/sistema_completo/tensao_saida_inversor_4.PNG}
		\caption{Tensões de saída sintetizadas para as fases A, B e C}
	\end{subfigure}
    \caption{Análise das tensões de saída sintetizadas após o filtro LCL}
    \label{fig:sim-tensao-saida-inversor}
\end{figure}

\begin{figure}[!hbt]
	\centering
	\begin{subfigure}[b]{\textwidth}
		\centering
		\includegraphics[width=\textwidth]{figuras/sim_figures/sistema_completo/corrente_saida_inversor_1.PNG}
		\caption{Corrente de saída na fase A}
    \end{subfigure}

    \begin{subfigure}[b]{\textwidth}
		\centering
		\includegraphics[width=\textwidth]{figuras/sim_figures/sistema_completo/corrente_saida_inversor_fft.PNG}
		\caption{Expectro em frequência da corrente na fase A. Os harmônicos de baixa frequência também sofreram uma boa atenuação}
    \end{subfigure}

	\begin{subfigure}[b]{\textwidth}
		\includegraphics[width=\textwidth]{figuras/sim_figures/sistema_completo/corrente_saida_inversor_2.PNG}
		\caption{Correntes de saída para as fases A, B e C}
	\end{subfigure}
    \caption{Análise das corrente de saída após o filtro LCL. Estes resultados foram obtidos após a inserção de corrente pelo conversor \textit{boost} no barramento CC}
    \label{fig:sim-corrente-saida-inversor}
\end{figure}

Na Fig. \ref{fig:sim-tensao-corrente-boost} pode-se verificar os valores de tensão e de corrente na entrada do conversor \textit{Boost}.
É possível ver que a tensão na entrada do conversor é aproximadamente 250 V, e a corrente de entrada aproximadamente 10 A, produzindo desta forma uma potência de entrada $P_{in}$ = 2500 W.

\begin{figure}[!hbt]
	\centering
	\begin{subfigure}[b]{\textwidth}
		\centering
		\includegraphics[width=\textwidth]{figuras/sim_figures/sistema_completo/tensao_entrada_boost_2.PNG}
		\caption{Tensão na entrada do conversor \textit{Boost}. A tensão média na entrada foi de 250 $V_{DC}$}
	\end{subfigure}
	
	\begin{subfigure}[b]{\textwidth}
		\centering
		\includegraphics[width=\textwidth]{figuras/sim_figures/sistema_completo/corrente_entrada_boost_2.PNG}
		\caption{Corrente medida na entrada do conversor \textit{Boost}. A corrente média foi de 10 A}
	\end{subfigure}

	\caption{Verificação da tensão e corrente de entrada no conversor \textit{Boost} para cálculo da potência de entrada do sistema}
    \label{fig:sim-tensao-corrente-boost}
\end{figure}	


\section{Resultados Experimentais}