\begin{anexosenv}

\partanexos

\chapter{Código em Python para o projeto do filtro LCL}\label{codigo-python-filtro}

\begin{lstlisting}[language=Python]
    import numpy as np 
    import matplotlib.pyplot as plot

    # tensao da rede
    Vn = 220
    # potencia da rede
    Pn = 3000
    # frequencia da rede
    fn = 60
    # frequencia de chaveamento
    fsw = 5000
    # impedancia de base
    Zb = (Vn*Vn)/Pn
    
    # capacitancia de base
    Cb = 1/(2*np.pi*fn*Zb)
    print("Zb: %lf" %Zb)
    print("Cb: %lf" %Cb)
    
    # porcentagem de ripple de corrente - 10 %
    ripple_i = 0.10
    
    # maximo ripple de corrente
    i_lmax = ripple_i*(Pn*np.sqrt(2))/Vn
    print("I_L_max: %lf A" %i_lmax)
    
    # indutor do lado do inversor
    L1 = Vn/(2*np.sqrt(6)*fsw*i_lmax)
    print("L1: %lf mH" %(L1*1e3))
    
    # capacitancia do filtro
    Cf = 0.05*Cb
    print("Cf: %lf uF" %(Cf*1e6))
    
    # ripple de corrente
    ka = 0.2
    a = L1*Cb*(2*np.pi*fsw)**2
    print("a: %lf" %a)

    r = np.arange(0.05,1,0.001)
    ka = 1/(np.absolute(1+r*(1-a*0.05)))
    plot.plot(r,ka)
    plot.xlabel("Fator r")
    plot.ylabel("Atenuacao Ka (%)")
    plot.scatter(0.164,0.2,s=100,color='r')
    plot.axhline(y=0.2,xmin=0,xmax=0.16400,color='g')
    plot.axvline(x=0.164,ymin=0,ymax=0.200,color='g')
    plot.savefig('plot.png')

    r = 0.05
    ka =  1/(np.absolute(1+r*(1-a*0.05)))
    while( ka > 0.2):
    r = r + 0.001
    ka =  1/(np.absolute(1+r*(1-a*0.05)))

    print("r: %lf" %r)
    # indutor do lado da rede
    L2 = r*L1
    print("L2: %lf mH" %(L2*1e3))
    
    # frequencia de ressonancia
    w_res = np.sqrt((L1+L2)/(L1*L2*Cf))
    f_res = w_res/(2*np.pi)
    print("10*fn: %lf" %(10*fn))
    print("f_res: %lf" %(f_res))
    print("f_sw/2: %lf" %(fsw/2))
    
    # resistencia de amortecimento
    Rf = 1/(3*w_res*Cf)
    print("Rf: %lf" %(Rf))

\end{lstlisting}

\end{anexosenv}

