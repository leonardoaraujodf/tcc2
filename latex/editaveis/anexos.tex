\begin{anexosenv}

\partanexos

\chapter{Código em Python para o projeto do filtro LCL}\label{codigo-python-filtro}

\begin{lstlisting}[language=Python,caption=Projeto do filtro LCL e controladores]
    import numpy as np 
    import matplotlib.pyplot as plot
    
    # tensao da rede
    Vn = 380
    # potencia da rede
    Pn = 10000
    # frequencia da rede
    fn = 60
    # frequencia de chaveamento
    fsw = 5000
    # impedancia de base
    Zb = (Vn*Vn)/Pn
    # capacitancia de base
    Cb = 1/(2*np.pi*fn*Zb)
    print("Zb: %lf" %Zb)
    print("Cb: %lf" %Cb)
    # porcentagem de ripple de corrente no lado do conversor - 25 %
    ripple_i = 0.25
    # maximo ripple de corrente
    i_lmax = ripple_i*(Pn*np.sqrt(2))/Vn
    print("i_l_max: %lf A" %i_lmax)
    # indutor do lado do inversor
    L1 = Vn/(2*np.sqrt(6)*fsw*i_lmax)
    print("L1: %lf mH" %(L1*1e3))
    # capacitancia do filtro
    Cf = 0.063*Cb
    print("Cf: %lf uF" %(Cf*1e6))
    
    
    # fator a
    a = L1*Cb*(2*np.pi*fsw)**2
    print("a: %lf" %a)

    # visualizacao da atenuacao ka em funcao do fator r
    r = np.arange(0.08,1.3,0.001)
    ka = 1/(np.absolute(1+r*(1-a*0.05)))
    plot.plot(r,ka)
    plot.xlabel("Fator r")
    plot.ylabel("Atenuacao Ka")
    # plot.scatter(0.164,0.2,s=100,color='r')
    # plot.axhline(y=0.2,xmin=0,xmax=0.16400,color='g')
    # plot.axvline(x=0.164,ymin=0,ymax=0.200,color='g')
    plot.savefig('plot.png')

    # valor escolhido de r
    r = 1.0
    ka =  1/(np.absolute(1+r*(1-a*0.05)))

    # while( ka > 0.2):
    #   r = r + 0.001
    # ka =  1/(np.absolute(1+r*(1-a*0.05)))

    # fator r
    print("r: %lf" %r)
    # fator de atenuacao
    print("ka: %lf" %ka)
    # indutor do lado da rede
    L2 = r*L1
    print("L2: %lf mH" %(L2*1e3))
    # frequencia de ressonancia
    w_res = np.sqrt((L1+L2)/(L1*L2*Cf))
    f_res = w_res/(2*np.pi)
    print("10*fn: %lf" %(10*fn))
    print("f_res: %lf" %(f_res))
    print("f_sw/2: %lf" %(fsw/2))
    # resistencia de amortecimento
    Rf = 1/(3*w_res*Cf)
    print("Rf: %lf" %(Rf))

    # calculo dos controladores
    Lf = L1 + L2
    print("Lf: %lf" %(Lf))
    Tm = 1/500
    T3 = Lf/Rf
    k3 = Lf/(2*Tm)
    print("T3: %lf" %T3)
    print("k3: %lf" %k3)

    # projeto dos controladores de corrente e da malha vcc
    print("\n--- Projeto dos controladores ---\n")

    # periodo de amostragem
    T = 1/10000

    # utilizar o metodo Tustin para discretizar os controladores e chegar nas equacoes analiticas abaixo

    a = (k3*(T + 2*T3))/(2*T3)
    b = (k3*(T - 2*T3))/(2*T3)

    print("\n--- Controlador PI de corrente ---\n")
    print(f"y[n] = y[n-1] + {a}*x[n] {b}*x[n-1]")

    Tm = 1/(500)
    C = 940e-6
    T4 = 8*Tm
    k4 = C/(4*Tm)
    print("T4: %lf" %T4)
    print("k4: %lf" %k4)

    T = 1/10000
    a = (k4*(T + 2*T4))/(2*T4)
    b = (k4*(T - 2*T4))/(2*T4)

    print("\n--- Controlador PI elo de corrente continua ---\n")
    print(f"y[n] = y[n-1] + {a}*x[n] {b}*x[n-1]")

\end{lstlisting}

\chapter{Código em C utilizado no DSP TMS320F28335}

Acesse o seguinte link para acessar o código no repositório do trabalho no github: \url{https://github.com/leonardoaraujodf/tcc2/blob/master/src/Inversor_Trifasico/main3.c}

\end{anexosenv}

