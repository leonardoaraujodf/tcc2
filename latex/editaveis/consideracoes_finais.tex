\chapter{Considerações Finais}

No presente trabalho, verificou-se a importância do estudo, implementação e utilização de fontes eólicas 
para a geração de energia elétrica. Visto que estas fontes são altamente dependentes de fatores meteorológicos, 
climáticos e sazonais, a integração destas com o sistema elétrico de potência pode ocasionar sérios distúrbios 
no funcionamento da rede elétrica, o que têm sido um tema de amplas pesquisas no ambiente acadêmico, 
como foi em especial, o objetivo primário de investigação deste trabalho.

Por conseguinte, detalhou-se alguns dos diversos componentes de uma bancada para acoplamento de unidades eólicas, 
sua modelagem matemática, discretização, projeto de componentes físicos e a implementação dos algoritmos de controle
e de sincronização em microcontrolador, tendo em vista que a teoria demonstrada possui alicerce teórico bem fundamentado.

Houve, ainda, a preocupação de que os procedimentos fossem de fácil reprodução para outros estudos de pesquisa e 
pela indústria. Em especial destaque, o projeto do filtro LCL e das malhas de controle e de sincronismo do inversor.
Desta forma, baseado nos procedimentos demonstrados, pode-se determinar a escolha do melhor filtro para acoplamento da 
unidade eólica à rede elétrica, de forma a atender tanto os requisitos físicos de projeto como de critérios 
estabelecidos pelos organismos governamentais. As equações discretizadas foram convertidas e implementadas em 
linguagem C, e utilizadas em um DSP criado para atendar as demandas da indústria de controladores de potência. Estas podem 
ser, ainda, adaptadas para diversos outros microcontroladores e DSPs.

Os resultados computacionais obtidos mostraram que o projeto do inversor trifásico atendeu satisfatoriamente os requisitos
impostos, de forma que houve o controle correto da tensão no barramento de corrente contínua para o caso da simulação de 
injeção de potência constante e variável pelo aerogerador no sistema e também para o caso em que este entrava em falta, 
requisitando potência da rede elétrica. Verificou-se ainda que o filtro LCL atenuava corretamente as harmônicas de alta 
frequência injetadas devido ao chaveamento dos IGBTs. Com isto, validou-se as malhas de controle e de sincronismo, 
assim como os componentes físicos utilizados, dando robustez para a fundamentação teórica demonstrada e prosseguimento 
com os testes experimentais. 

Para os resultados obtidos em laboratório, pode-se realizar o ajuste das placas de aquisição e condionamento, validar a malha de sincronismo 
através do algoritmo do DSOGI-FLL, verificar e atestar o funcionamento da implementação do SPWM através do módulos 
disponibilizados pelo DSP e atestar o funcionamento das malhas internas de controle de corrente, para o caso em que o inversor 
estava com sua saída isolada da rede elétrica. 
Foram realizados experimentos com o controle do barramento de corrente contínua, e a princípio obteve-se sucesso na 
verificação deste, mas os resultados não foram apresentados neste trabalho pois ainda não se conseguiu verificar o 
funcionamento do inversor na operação de STATCOM. Posteriormente, outros testes experimentais serão 
realizados para verificação do inversor nesta última operação e também o teste controle da injeção de potência 
no barramento através do conversor \textit{boost}.

Com isto, o presente trabalho apresenta somente uma estimativa inicial no que tange o
desenvolvimento prático de estudos de bancadas eólicas e servirá como ponto de apoio e
alicerce para que outros pesquisadores e entusiastas realizem o desenvolvimento 
de estudos futuros de controladores de potência no que tange, por exemplo, 
a determinação de distorções harmônicas, máquinas síncronas, aerogeradores, 
projeto de filtros, algoritmos de controle e etc. 