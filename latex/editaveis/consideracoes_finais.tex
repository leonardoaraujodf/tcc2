\chapter{Conclusões}

No presente trabalho verificou-se a importância do estudo, implementação e utilização de fontes eólicas 
para a geração de energia elétrica. Observa-se que estas fontes são altamente dependentes de fatores meteorológicos, 
climáticos e sazonais, de forma que integração destas com o sistema elétrico de potência podem ocasionar distúrbios 
no funcionamento da rede elétrica. 
O estudo destes distúrbios e formas de atenuação foram o objetivo primário de investigação deste trabalho.

Por conseguinte, detalhou-se alguns dos diversos componentes de uma bancada para acoplamento de unidades eólicas, 
sua modelagem matemática, discretização, projeto de componentes físicos e a implementação dos algoritmos de controle
e de sincronização em microcontrolador.

Houve uma preocupação de que os procedimentos fossem de fácil reprodução para outros estudos de pesquisa e 
pela indústria. Em especial destaque, o projeto do filtro LCL e das malhas de controle e de sincronismo do inversor.
Com isto pode-se projetar o filtro para acoplamento da  unidade eólica à rede elétrica de forma a atender 
tanto os requisitos físicos de projeto como de critérios estabelecidos pelos organismos governamentais. 
As equações discretizadas foram convertidas e implementadas em linguagem C, e utilizadas em um DSP projetado 
para atendar as demandas da indústria de controladores de potência. 
Estas equações podem ser ainda adaptadas para diversos outros microcontroladores e DSPs.

Os resultados de simulação computacional mostraram que o projeto do inversor trifásico atendeu os requisitos
impostos, de forma que houve o controle da tensão no barramento de corrente contínua dentro do limite de tensão CC estabelecido,
para o caso da simulação de injeção de potência constante pelo aerogerador, para o caso da injeção de potência variável pelo aerogerador 
e ainda para o caso em que o aerogerador entrava em falta, fazendo com que o inversor requisitasse potência da rede elétrica. 

Os resultados de simulação computacional revelaram ainda que o filtro LCL atenuava as harmônicas de alta frequência dentro 
das especificações projetadas, harmônicas estas devidas ao chaveamento dos IGBTs.
Desta forma, validou-se as malhas de controle e de sincronismo, assim como os componentes físicos utilizados.

Para os resultados obtidos em laboratório, pode-se realizar o ajuste das placas de aquisição e condionamento, validar a malha de sincronismo 
através do algoritmo do DSOGI-FLL, verificar e atestar o funcionamento da implementação do SPWM através do módulos 
disponibilizados pelo DSP e atestar o funcionamento das malhas internas de controle de corrente, para o caso em que o inversor 
estava com sua saída isolada da rede elétrica. 

\section{Desenvolvimentos Futuros}

Testes ainda estão sendo realizados para verificação do controle de tensão do barramento de corrente contínua, e até o momento 
da publicação do presente trabalho observou-se que para o caso em que o inversor estava em atuando isolado da rede elétrica, 
o sistema atingiu os requisitos impostos.

Apesar da validação da malha de corrente contínua do inversor, não foi apresentado os resultados obtidos neste trabalho
pois ainda não foi possível atestar o funcionamento do inversor na operação de STATCOM. 

Posteriormente outros testes experimentais serão realizados para verificação do inversor na operação de STATCOM e também 
para o controle da injeção de potência no barramento através do conversor \textit{boost}.

Baseado no que foi obtido e nos desenvolvimentos planejados deste trabalho, é possível dar prosseguimento a trabalhos relacionados a 
determinação de distorções harmônicas, máquinas síncronas, aerogeradores, projeto de filtros, algoritmos de controle e etc.