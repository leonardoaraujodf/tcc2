\begin{resumo}
 
É crescente a demanda de energia no mundo e o uso de fontes de energias alternativas tem tido destaque 
no contexto nacional e internacional, em especial a energia eólica, a qual tem sido alvo de constantes 
investimentos de instituições públicas e privadas. No Brasil, esta fonte de energia tem 
papel relevante em sua matriz elétrica e 
é há um crescimento no número de estudos de acesso de parques eólicos ao sistema elétrico de potência. 
Devido a variabilidade da potência produzida por estas fontes, aspectos que envolvem a qualidade de
energia elétrica são alvos de estudos para a operação destes sistemas, com o emprego de conversores de 
potência, filtros de acoplamento e sistemas de controle.
O objetivo deste trabalho é a investigação através de simulações computacionais e testes laboratoriais
do acoplamento de unidades eólicas à rede elétrica. 
Este faz utilização de um modelo computacional em \textit{software} para validação dos componentes 
físicos projetados e algoritmos de controle do sistema e de uma bancada laboratorial empregando um
inversor trifásico para atestar o funcionamento do modelo.
As simulações computacionais conduzidas em \textit{software} demonstram que a tensão no barramento 
de corrente contínua é fixada em um determinado valor pelas malhas de controle implementadas 
realizando o controle do fluxo de potência. Com os testes laboratoriais é possível observar o funcionamento
dos diversos subsistemas que compõem o inversor trifásico.
A contribuição deste trabalho se dá pela montagem de uma bancada eólica. Este fato possibilita estudos 
referentes à conexão de unidades eólicas e de outros sistemas que empregam inversores trifásicos à rede.

%    O resumo deve ressaltar o objetivo, o método, os resultados e as conclusões 
% do documento. A ordem e a extensão
% destes itens dependem do tipo de resumo (informativo ou indicativo) e do
% tratamento que cada item recebe no documento original. O resumo deve ser
% precedido da referência do documento, com exceção do resumo inserido no
% próprio documento. (\ldots) As palavras-chave devem figurar logo abaixo do
% resumo, antecedidas da expressão Palavras-chave:, separadas entre si por
% ponto e finalizadas também por ponto. O texto pode conter no mínimo 150 e 
% no máximo 500 palavras, é aconselhável que sejam utilizadas 200 palavras. 
% E não se separa o texto do resumo em parágrafos.

 \vspace{\onelineskip}
    
 \noindent
 \textbf{Palavras-chaves}: inversor trifásico, energia eólica, sistema elétrico de potência, qualidade de energia, conversores de potência, SPWM, Conversor \textit{Boost}.
\end{resumo}
