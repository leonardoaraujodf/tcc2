\begin{resumo}
 
É crescente a demanda de energia no mundo, e o uso de fontes de energias alternativas tem tido destaque no contexto 
internacional, em especial a energia eólica, a qual tem sido alvo de constantes investimentos, tanto de organismos 
governamentais quanto na área privada. No Brasil, esta fonte de energia tem papel relevante em sua matriz energética e 
é de especial interesse o estudo de acoplamento de unidades eólicas à sistemas elétricos de potência. 
Devido ao aspecto variável do fornecimento de energia da fonte em questão, a área da qualidade da energia elétrica 
é um objeto de estudo para operação destes sistemas, fazendo uso do projeto de conversores de potência, filtros de 
acoplamento e algoritmos de controle e sincronismo. Desta forma, o objetivo deste trabalho foi a simulação computacional 
e a elaboração de uma bancada para estudo e investigação do acoplamento de unidades eólicas à rede elétrica, de forma 
a apresentar a teoria de projeto de inversores trifásicos de potência, filtros LCL de acoplamento, 
malhas de controle de corrente e de controle de tensão do barramento de corrente contínua e malha de sincronia. Realizou-se a 
simulação computacional do sistema no \textit{software} PSIM e obteu-se resultados satisfatórios para operação 
do inversor no que tange o controle do barramento de corrente contínua através da injeção ou recebimento de potência da 
rede elétrica. Logo após, realizou-se experimentos práticos com a bancada, obtendo-se resultados 
convincentes de funcionamento para os diversos subsistemas que compõem o inversor trifásico. 
A bancada servirá ainda como base para projetos e estudos futuros de investigação do acoplamento 
de unidades eólicas à rede elétrica.

%    O resumo deve ressaltar o objetivo, o método, os resultados e as conclusões 
% do documento. A ordem e a extensão
% destes itens dependem do tipo de resumo (informativo ou indicativo) e do
% tratamento que cada item recebe no documento original. O resumo deve ser
% precedido da referência do documento, com exceção do resumo inserido no
% próprio documento. (\ldots) As palavras-chave devem figurar logo abaixo do
% resumo, antecedidas da expressão Palavras-chave:, separadas entre si por
% ponto e finalizadas também por ponto. O texto pode conter no mínimo 150 e 
% no máximo 500 palavras, é aconselhável que sejam utilizadas 200 palavras. 
% E não se separa o texto do resumo em parágrafos.

 \vspace{\onelineskip}
    
 \noindent
 \textbf{Palavras-chaves}: inversor trifásico. energia eólica. sistema elétrico de potência. qualidade de energia. conversores de potência. SPWM. Conversor \textit{Boost}.
\end{resumo}
